\documentclass[12pt]{article}
% \documentclass[12pt, oneside]{book}
\usepackage{mypkg}
% "ming": 新細明體, "kai": TW-Kai/BiauKaiTC, "serif": Noto Serif CJK TC(有襯線)/Songti TC, "sans": Noto Sans CJK TC(無襯線)/PingFang TC
% Overleaf沒有新細明體
% \setMyCJK{sans}
\setCJKmainfont{PingFang TC}
\usepackage{newtxtext, newtxmath}

\title{簡單模板}
\author{作者}
\date{\today}

\begin{document}
\maketitle
\tableofcontents

\section{中文測試}
\zhlipsum[1]
\section{英文測試}
\lipsum[1]
\section{數學測試}
\begin{align}
	\int_{-\pi }^{\pi } x^{4} + \frac{3}{7}x^2 + \sqrt{6}   \,\mathrm{d}x & = \frac{2 \pi (5 \pi^{2} + 35 \sqrt{6} + 7 \pi^{4})}{35} \\
\end{align}
\section{程式碼測試}
\subsection{pseudocode code}
\begin{algorithm}
	\caption{演算法名稱}
	\begin{algorithmic}[1]
		\State \(target \gets \text{RANDOM}(1, 100) \)
		\State \(attempts \gets 0\)
		\While{True}
		\State \(guess \gets \text{INPUT}() \)
		\State \(attempts \gets attempts + 1\)
		\If{\(guess = target\)}
		\State \textbf{break}
		\ElsIf{\(guess > target\)}
		\State \textbf{print} "Too high"
		\Else
		\State \textbf{print} "Too low"
		\EndIf
		\EndWhile
	\end{algorithmic}
\end{algorithm}

\subsection{Python程式碼}
\begin{minted}{python}
    print("Hello, world!")
    def add(a, b):
        return a + b

    class MyClass:
        def __init__(self, name):
            self.name = name

        def say_hello(self):
            print(f"Hello, {self.name}!")

    my_instance = MyClass("Alice")
    my_instance.say_hello()
    print("Hi")
\end{minted}

\section{Color Box test}
\dfn{Limit in R}{Let \(\{s_n\}\) be a sequence in \(\mathbb{R}\) }
\thm{}{}
\ex{這是範例標題}{
	\[
		\int_{-\pi }^{\pi } x^{4} + \frac{3}{7}x^2 + \sqrt{6}   \,\mathrm{d}x = \frac{2 \pi (5 \pi^{2} + 35 \sqrt{6} + 7 \pi^{4})}{35}
	\]
}
\nt{這是一個Note}
\qs{問題}{這是一個問題}
這是一個註釋\footnote{這是一個註釋}。

% create 3 column figure with 3 images example
\begin{figure}[H]
	\centering
	\begin{subfigure}{0.3\textwidth}
		\centering
		\includegraphics[width=0.9\linewidth]{example-image-a}
		\caption{Caption 1}
		\label{fig:sub1}
	\end{subfigure}
	\begin{subfigure}{0.3\textwidth}
		\centering
		\includegraphics[width=0.9\linewidth]{example-image-b}
		\caption{Caption 2}
		\label{fig:sub2}
	\end{subfigure}
	\begin{subfigure}{0.3\textwidth}
		\centering
		\includegraphics[width=0.9\linewidth]{example-image-c}
		\caption{Caption 3}
		\label{fig:sub3}
	\end{subfigure}
	\caption{Caption for this figure with three images}
	\label{fig:image3}
\end{figure}


% 參考文獻
\nocite{*}
\bibliographystyle{plain}
\bibliography{references}

\end{document}