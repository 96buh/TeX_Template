\documentclass[12pt]{article}
\usepackage{customstyle}
% "ming": 新細明體, "kai": TW-Kai(楷體), "serif": Noto Serif CJK TC(有襯線), "sans": Noto Sans CJK TC(無襯線)
% Overleaf沒有新細明體
\setMyCJK{kai}
\usepackage{newtxtext, newtxmath}

\title{簡單模板}
\author{作者}
\date{\today}

\begin{document}
\maketitle

\section{中文測試}
\zhlipsum[1]
\section{英文測試}
\lipsum[1]
\section{數學測試}
\begin{align}
	\int_{-\pi }^{\pi } x^{4} + \frac{3}{7}x^2 + \sqrt{6}   \,\mathrm{d}x & = \frac{2 \pi (5 \pi^{2} + 35 \sqrt{6} + 7 \pi^{4})}{35} \\
\end{align}
\[
	\begin{bmatrix}
		1 & 2 & 3 \\
		4 & 5 & 6 \\
		7 & 8 & 9
	\end{bmatrix}
\]

\section{程式碼測試}

\begin{minted}{python}
# Vanilla gradient descent
    w = initialize_weights()
    for t in range(num_steps):
        dw = compute_gradient(loss_fn, data, w)
        w -= learning_rate * dw
\end{minted}

\section{}


% 參考文獻
\nocite{*}
\bibliographystyle{plain}
\bibliography{references}

\end{document}